% KOMA-Script Klasse 'scrreprt' inkl. Einstellungen
\documentclass[
	pdftex,
	a4paper,
	oneside,
	12pt,
	% Tabellen- und Abbildungsverzeichnis im Inhaltsverzeichnis anzeigen
	listof=numbered,
	% Litaraturverzeichnis im Inhaltsverzeichnis
	bibliography=totocnumbered
]{scrreprt}

% Ändern der Seitenränder
\usepackage[a4paper, left=3cm, right=2cm, top=2.5cm, bottom=3.5cm]{geometry}

% Zeichencodierung und Zeichensatz
\usepackage[utf8]{inputenc}
\usepackage[T1]{fontenc}

% Verwendung der neuen deutschen Rechtschreibung
\usepackage[ngerman]{babel}

% Verwendung von schweizer Anführungszeichen
\usepackage[style=swiss]{csquotes}

% Umlaute in Bibtex
%\usepackage{bibgerm} todo: replace with babelbib
% Bibtex Style
\bibliographystyle{plain}

% Typografische Feinheiten
\usepackage[final]{microtype}

% Schriftart
\usepackage{lmodern}

% Absätze durch Leerraum trennen
\usepackage{parskip}

% Zeilenabstand 1.5
\usepackage{setspace}
\onehalfspacing

% Einbindung von Grafiken ermöglichen
\usepackage{graphicx}
% Standard Pfad für Grafiken
\graphicspath{{images/}}

% Verwendung von Farben ermöglichen
\usepackage[usenames,dvipsnames]{color}

% Verwendung von PDF Features
\usepackage{hyperref}
% PDF Links aktivieren
\hypersetup{
	colorlinks=true,
	linkcolor=Black,
	citecolor=Black,
	filecolor=Black,
	menucolor=Black,
	urlcolor=Black,
	bookmarksnumbered=true
}
% Link springt auf Bild anstatt Beschriftung
\usepackage[all]{hypcap}

% Inkludieren von Sourcecode Files
\usepackage{listings}
\renewcommand{\lstlistlistingname}{Quellcodeverzeichnis}
\definecolor{LineNumberColor}{rgb}{0.8,0.8,0.8}
\lstset{
	breaklines=true,
	numbers=left,
	tabsize=2,
	basicstyle=\footnotesize\ttfamily,
	numberstyle=\color{LineNumberColor}\ttfamily,
	frame=bottomline
}
% Syntax Highlighting für Java
\definecolor{JavaStringColor}{rgb}{0.6,0,0}
\definecolor{JavaCommentColor}{rgb}{0.25,0.5,0.35}
\definecolor{JavaKeywordColor}{rgb}{0.5,0,0.35}
\definecolor{JavaDocColor}{rgb}{0.25,0.35,0.75}
\lstdefinestyle{customjava}{
	language=Java,
	keywordstyle=\bfseries\color{JavaKeywordColor},
	stringstyle=\color{JavaStringColor},
	commentstyle=\color{JavaCommentColor},
	morecomment=[s][\color{JavaDocColor}]{/**}{*/},
	showstringspaces=false,
}

% Erlaubt einfachere Berechnungen
\usepackage{calc}

% Erweiterte Überschriften von Bildern und Listings
\usepackage{caption}
% Überschrift für Listings neu definieren
\DeclareCaptionFont{whitefont}{\color{white}}
\DeclareCaptionFormat{listing}{\colorbox{black}{\parbox{\textwidth-2\fboxsep}{\hspace{5pt}#1#2#3}}}
\captionsetup[lstlisting]{format=listing,labelfont=whitefont,textfont=whitefont, singlelinecheck=false, margin=0pt, font={bf,footnotesize}}
% Hinweis zur Breite der Caption: \colorbox fügt an den Aussenseiten Space der Länge \fboxsep ein, wodurch die \parbox mit Länge \textwidth zu breit wird. Deshalt wird von \textwidth zwei mal \fboxsep abgezogen.

% Durchstreichen von Text
\usepackage{ulem}

% Einbindung von Blindtext
\usepackage{blindtext}